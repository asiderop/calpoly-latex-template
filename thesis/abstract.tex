
This is where the abstract goes.  Hopefully this document will serve as an example for preparing a Cal Poly Master's thesis.  It was thrown together pretty quickly.  A lot more neat LaTeX features, help, and examples can be found on the web.  Here is one: http://en.wikibooks.org/wiki/LaTeX

For developing LaTeX documents in the Windows environment, I use TeXnicCenter (http://www.toolscenter.org/).  A simple WYSIWYG LaTeX editor (though I had problems getting it to work with this thesis format) is \textbf{LyX} (http://www.lyx.org/).

I use \textbf{InkScape} (http://www.inkscape.org/) to create any drawings/figures needed.  It is a free vector graphics editor that is very powerful and popular.  There is an example figure produced with InkScape in Figure~\ref{fig:inkscape-example}.  InkScape can export images in many different formats.  Export your images as PDF or EPS and put into your LaTeX document.  If you're creating a PDF document with \textbf{pdflatex}, then export as a PDF image.  If you're creating PostScript then export as EPS.  Rasterized images such as JPEG can also be easily included in LaTeX.

LaTeX can also produce nice equations.  Did you know that $\sum_{n=0}^{\infty} \frac{(-1)^n}{2n+1} = \frac{1}{1} - \frac{1}{3} + \frac{1}{5} - \frac{1}{7} + \frac{1}{9} - \cdots = \frac{\pi}{4}$ ?  A non-inline equation can be found in Figure~\ref{eqn:example}.  I treated my equations as figures but they can be treated specially as Equations.

An example of a table can be found in Table~\ref{table:performance}.

The bibliography section is very easy to create.  When gathering references, I used the ACM digital library (http://portal.acm.org/portal.cfm) to grab the Bibtex entries.  Papers in the digital library have Bibtex entries ready to be copied and pasted into your bibliography.  Create a separate file called something like ``bibliography.bib'' and paste in your Bibtex entries.  LaTeX (and Bibtex) generate your bibliography section for you -- very easy! I can cite references very easily.  Here is a paper called \emph{Dual contouring of hermite data}~\cite{DualContouring}.  Here is a paper called \emph{Surface simplification using quadric error metrics}~\cite{QuadricErrorMetrics}.  I've also cited software located at some websites \cite{NormalMapper}~\cite{nVidiaMelody}.

