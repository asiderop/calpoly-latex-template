
Here is the results section.

LaTeX is a document markup language and document preparation system for the TeX typesetting program.

It is widely used by mathematicians, scientists, philosophers, engineers, and scholars in academia and the commercial world, and by others as a primary or intermediate format (e.g. translating DocBook and other XML-based formats to PDF) because of the quality of typesetting achievable by TeX. It offers programmable desktop publishing features and extensive facilities for automating most aspects of typesetting and desktop publishing, including numbering and cross-referencing, tables and figures, page layout and bibliographies.

LaTeX is intended to provide a high-level language to access the power of TeX. LaTeX essentially comprises a collection of TeX macros, and a program to process LaTeX documents. Since TeX's formatting commands are very low-level, it is usually much simpler for end-users to use LaTeX.

LaTeX was originally written in 1984 by Leslie Lamport at SRI International and has become the dominant method for using TeX—few people write in plain TeX anymore.

LaTeX is based on the idea that authors should be able to focus on the meaning of what they are writing, without being distracted by the visual presentation of the information. In preparing a LaTeX document, the author specifies the logical structure using familiar concepts such as chapter, section, table, figure, etc., and lets the LaTeX system worry about the presentation of these structures. It therefore encourages the separation of layout from content, while still allowing manual typesetting adjustments where needed. This is similar to the mechanism by which many word processors allow styles to be defined globally for an entire document, or the CSS mechanism used by HTML.

LaTeX can be arbitrarily extended by using the underlying macro language to develop custom formats. Such macros are often collected into packages which are available to address special formatting issues such as complicated mathematical content or graphics. In addition, there are numerous commercial implementations of the entire TeX system, including LaTeX, to which vendors may add extra features like additional typefaces and telephone support. LyX is a free visual document processor that uses LaTeX for a back-end. TeXmacs is a free, WYSIWYG editor with similar functionalities as LaTeX, but a different typesetting engine.

A number of popular commercial desktop publishing systems use modified versions of the original TeX typesetting engine. The recent rise in popularity of XML systems and the demand for large-scale batch production of publication-quality typesetting from such sources has seen a steady increase in the use of LaTeX.


\begin{table}
\begin{center}

\begin{tabular}{|c|c|c|c|c|c|c|}
\hline 
&
\multicolumn{2}{c|}{Some Data}&
&
\multicolumn{2}{c|}{Some More Data}&
\tabularnewline
\hline
\hline 
&  Hi-Res&  Lo-Res&  Reduction&  Hi-Res&  Lo-Res&  Speedup
\tabularnewline
\hline 
Row Data A &  225,467&  43,850&  80.6\%&  360&  90&  4.0
\tabularnewline
\hline 
Row Data B &  225,467&  16,388&  92.7\%&  360&  26&  13.8
\tabularnewline
\hline 
\end{tabular}


\captionfonts
\caption[Performance data]{Here is some performance data for the system.}
\label{table:performance}
\end{center}
\end{table}


LaTeX is a document markup language and document preparation system for the TeX typesetting program.

It is widely used by mathematicians, scientists, philosophers, engineers, and scholars in academia and the commercial world, and by others as a primary or intermediate format (e.g. translating DocBook and other XML-based formats to PDF) because of the quality of typesetting achievable by TeX. It offers programmable desktop publishing features and extensive facilities for automating most aspects of typesetting and desktop publishing, including numbering and cross-referencing, tables and figures, page layout and bibliographies.

LaTeX is intended to provide a high-level language to access the power of TeX. LaTeX essentially comprises a collection of TeX macros, and a program to process LaTeX documents. Since TeX's formatting commands are very low-level, it is usually much simpler for end-users to use LaTeX.

LaTeX was originally written in 1984 by Leslie Lamport at SRI International and has become the dominant method for using TeX—few people write in plain TeX anymore.

LaTeX is based on the idea that authors should be able to focus on the meaning of what they are writing, without being distracted by the visual presentation of the information. In preparing a LaTeX document, the author specifies the logical structure using familiar concepts such as chapter, section, table, figure, etc., and lets the LaTeX system worry about the presentation of these structures. It therefore encourages the separation of layout from content, while still allowing manual typesetting adjustments where needed. This is similar to the mechanism by which many word processors allow styles to be defined globally for an entire document, or the CSS mechanism used by HTML.

LaTeX can be arbitrarily extended by using the underlying macro language to develop custom formats. Such macros are often collected into packages which are available to address special formatting issues such as complicated mathematical content or graphics. In addition, there are numerous commercial implementations of the entire TeX system, including LaTeX, to which vendors may add extra features like additional typefaces and telephone support. LyX is a free visual document processor that uses LaTeX for a back-end. TeXmacs is a free, WYSIWYG editor with similar functionalities as LaTeX, but a different typesetting engine.

A number of popular commercial desktop publishing systems use modified versions of the original TeX typesetting engine. The recent rise in popularity of XML systems and the demand for large-scale batch production of publication-quality typesetting from such sources has seen a steady increase in the use of LaTeX.

